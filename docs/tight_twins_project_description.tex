\documentclass[11pt,a4paper]{article}

\usepackage[polish]{babel}
\usepackage[utf8]{inputenc}
\usepackage[T1]{fontenc}
\usepackage{lmodern}
\usepackage{indentfirst}
\usepackage{fullpage}
\usepackage{enumerate}
\usepackage{graphicx}
\usepackage{authblk}

\selectlanguage{polish}

% Zdefiniowanie autorów i~tytułu:
\author{Jakub Dutkowski}
\author{Karol Dzitkowski}
\author{Tomasz Janiszewski}
\affil{Wydział Matematyki i Nauk Informacyjnych, Politechnika Warszawska}
\title{
	Kombinatoryka na Słowach\\
	Gra w unikanie ciasnych bliźniaków\\
	Opis projektu
	6b\footnote{Gra komputer kontra człowiek (co najmniej 2 poziomy trudności)}
 }
 
 \renewcommand\Authands{ i }

\begin{document}
\maketitle\
\newpage\


%Treść:
\section{Temat projektu}
Projekt będzie realizował grę w unikanie ciasnych bliźniaków,
w której biorą udział 2 osoby 
\begin{enumerate}
	\item\ Gracz Pierwszy (A) [sterowany przez maszynę]
	\item\ Gracz Drugi (B) [sterowany przez człowieka]
\end{enumerate}
Planszą jest słowo $s$, które składa się ze skończonej,
liczby znaków $n$ nad skończonym alfabetem $A$.
Gracz A wskazuje miejsce pomiędzy symbolami w słowie $s$, w którym gracz B
wstawia dowolny symbol z alfabetu $A$.
Zadaniem gracza A jest zmuszenie gracza B do stworzenia 
słowa zawierającego ciasnych bliźniaków.

\section{Aplikacja}
Aplikacja będzie składała się z dwóch widoków:
\begin{itemize}
 \item\ menu
 \item\ gra
\end{itemize}

\subsection{Widok menu}
W tym widoku będą znajdowały się elementy umożliwiające:

\begin{itemize}
	\item\ rozpoczęcie gry
	\item\ ustawienie długości słów (rozgrywki)
	\item\ ustawiania rozmiaru alfabetu $A$ (jako podzbioru alfabetu łacińskiego)
	\item\ wybór poziomu trudności
\end{itemize}

\subsection{Widok gry}
Po rozpoczęciu rozgrywki pojawia się słowo $s$,
składające się z jednego losowego
symbolu z wybranego alfabetu $A$.
Rozpoczyna gracz A, wskazując miejsce w słowie $s$,
w którym gracz B wstawia wybrany przez
siebie symbol z alfabetu $A$.
Gracze powtarzają czynności tak długo aż słowo $s$
\begin{enumerate}
	\item\ będzie zawierało ciasnych bliźniaków \label{wygrana:a}
	\item\ osiągnie maksymalną długość $n$ \label{wygrana:b}
\end{enumerate}

W przypadku \ref{wygrana:a} wygrywa gracz A, natomiast
w przypadku \ref{wygrana:b} wygrywa gracz B.
Po zakończeniu gry wyświetlany jest komunikat o jej wyniku.
W przypadku \ref{wygrana:a} w słowie $s$ podświetlani są
ciaśni bliźniacy.

\end{document}