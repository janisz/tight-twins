\documentclass{beamer}

\usepackage[utf8]{inputenc}
\usepackage[T1]{fontenc}
\usepackage{amsthm}
\usepackage{amssymb}
\usepackage{amsfonts}
\usepackage{amsmath}
\usepackage[]{algorithmicx}
\usepackage{algpseudocode}% http://ctan.org/pkg/algorithmicx
\usepackage[polish]{babel}
\selectlanguage{polish}
%%% fix for \lll
\let\babellll\lll
\let\lll\relax
\usetheme{Warsaw}

\title{
	Gra w unikanie ciasnych bliźniaków\\
}

\author{
	Jakub Dutkowski \and
	Karol Dzitkowski \and
	Tomasz Janiszewski \and
	Michał Padzik
}

\institute[MiniPW] % (optional)
{
  Wydział Matematyki i Nauk Informacyjnych\\ Politechnika Warszawska
}

\date[VLC 2013] % (optional)
{Kombinatoryka na słowach, Czerwiec 2016}

\logo{\includegraphics[height=1.0cm]{img/mini-logo.png}}

\begin{document}

\frame{\titlepage}
\begin{frame}
\tableofcontents
\end{frame}

\section{Opis Gry}

\begin{frame}
\frametitle{Temat}
	\begin{itemize}
		\item Gra komputer kontra człowiek (co najmniej 2 poziomy trudności)
		\item Gra komputer kontra komputer (co najmniej 2 poziomy trudności)
	\end{itemize}
\end{frame}

\begin{frame}
\frametitle{Definicje}
	\begin{block}{Bliźniaki}\label{def:blizniak}
Niech
$S = s_1 s_2 s_3 \ldots\ s_n$
będzie skonczoną sekwencją symboli. Dwie pod-sekwencje
$S_1 = s_{i_1} s_{i_2} s_{i_3} \ldots\ s_{i_k}$
i
$S_2 = s_{j_1} s_{j_2} s_{j_3} \ldots\ s_{j_k}$
nazywane są bliźniakami jeśli są identyczne oraz rozłączne, czyli
$$
\left\{ \begin{array}{ll}
s_{i_p} = s_{j_p} &\forall p = {1, 2, 3, \ldots\, k} \\
i_p \neq j_p &\forall p,q = {1, 2, 3, \ldots\, k}
\end{array} \right.
$$
	\end{block}
\end{frame}

\begin{frame}
\frametitle{Definicje}
	\begin{block}{Ciasne Bliźniaki}\label{def:ciasne_blizniak}
Mówimy, że para bliźniaków jest ciasna,
jeżeli połączenie zbiorów ich indeksów tworzy pełny segment
liczb całkowitych,
czyli para tworzy jeden blok.
	\end{block}
\end{frame}

\begin{frame}
\frametitle{Definicje}
	\begin{examples}{Przykład}
Dla przykładu ciąg
$23\overline{12}\underline{1}\overline{3}\underline{23}1$
ma ciasnych bliźniaków
w formie $123$, natomiast ciąg $123132312$ nie ma żadnych.
	\end{examples}
\end{frame}

\begin{frame}
\frametitle{Definicje}
Gra odbywa się na planszy -- słowie. Początkowo nie ma na niej nic.
Gracz pierwszy wskazuje miejsce w słowie w którym należy dopisać znak.
Gracz drugi dopisuje znak z alfabetu. Gracz pierwszy wygrywa
gdy w słowie znajdą się ciasne bliźniaki.
Gracz drugi wygrywa gdy w gra się zakończy i w słowie nie będzie ciasnych
bliźniaków.
Gra kończy się gdy w słowie są ciasne bliźniaki lub
został osiągnięty limit długości słowa.
\end{frame}

\section{Algorytmy}

\begin{frame}
\frametitle{Sprawdzanie czy sekwencja zawiera ciasnych bliźniaków}

\begin{algorithmic}
\State $S \gets $ sekwencja wejściowa
\State $n \gets $ długość $S$
\ForAll{$i \in 0\dots l$}
	\ForAll{$j \in 0\dots n-i$}
		\State $s \gets S[j \dots i]$
		\If {$s$ jest ciasnym bliźniakiem}
			\State
		    \Return $true$
	    \EndIf
	\EndFor
\EndFor
\State
\Return $false$
\end{algorithmic}
\end{frame}

\begin{frame}
\frametitle{Sprawdzanie czy sekwencja jest ciasnym bliźniakiem}

\begin{algorithmic}
\State $S \gets $ sekwencja wejściowa
\State $subsets \gets $ wszystkie równoliczne podziały $S$ na dwie sekwencje
\ForAll{$(a,b) \in subests$}
		\If {$a = b$}
			\State
		    \Return $true$
	    \EndIf
\EndFor
\State
\Return $false$
\end{algorithmic}

\end{frame}

\begin{frame}
\begin{algorithmic}
\State $subsets \gets \emptyset$
\ForAll{$i \in 0\dots 2^n$}
		\State $b \gets $ binarna reprezentacja $i$
		\State $ones \gets \sum b $ \Comment{liczba jedynek w $b$}
		\If {$ones = \frac{n}{2}$}
			\State $A \gets \emptyset$
			\State $B \gets \emptyset$
			\ForAll{$i \in 0 \dots n$}
				\If {$ones[i] = 1$}
					\State
					\Call{Append}{$A, S[i]$}
				\Else
					\State
					\Call{Append}{$B, S[i]$}
				\EndIf
			\EndFor
			\State $subsets \gets subsets ~ \cup ~ (A,B)$
	    \EndIf
\EndFor
\State
\Return $subsets$
\end{algorithmic}

\end{frame}

\subsection{MinMax}

\begin{frame}
\frametitle{MinMax}

Gracz pierwszy stara się zakończyć grę z jak najkrótszym słowem.
Natomiast gracz drugi z jak najdłuższym.

\end{frame}

\begin{frame}
\frametitle{MinMax}

\begin{algorithmic}
\Function{MaxMove}{$S$}
	\State $r \gets -\infty$
	\ForAll{możliwy ruch}
			\State $m \gets -\infty$
			\If {ruch nie powoduje powstanie ciasnych bliźniaków}
				\If {osiągnięto limit słowa}
					\State $m \gets +\infty$ \Comment{Wygrywamy}
				\Else
					\State $m \gets $ \Call{MinMove}{plansza po ruchu}
				\EndIf
			\Else
				\State $m \gets $ długość słowa \Comment{Chcemy grać jak najdłużej}
			\EndIf
			\State $r \gets $ \Call{$max$}{$r, m$}
	\EndFor
	\State
	\Return ruch z wartością $r$
\EndFunction
\end{algorithmic}

\end{frame}

\begin{frame}
\frametitle{MinMax}

\begin{algorithmic}
\Function{MinMove}{$S$}
	\State $r \gets +\infty$
	\ForAll{możliwy ruch}
			\State $m \gets +\infty$
			\If {ruch powoduje powstanie ciasnych bliźniaków}
				\If {osiągnięto limit słowa}
					\State $m \gets -\infty$ \Comment{Wygrywamy}
				\Else
					\State $m \gets $ \Call{MaxMove}{plansza po ruchu}
				\EndIf
			\Else
				\State $m \gets $ długość słowa \Comment{Chcemy grać jak najkrócej}
			\EndIf
			\State $r \gets $ \Call{$min$}{$r, m$}
	\EndFor
	\State
	\Return ruch z wartością $r$
\EndFunction
\end{algorithmic}

\end{frame}

\section{Optymalizacje}

\begin{frame}
\frametitle{Sprawdzanie czy sekwencja zawiera ciasnych bliźniaków}

\begin{algorithmic}
\State $S \gets $ sekwencja wejściowa
\State $n \gets $ długość $S$
\ForAll{$i \in 0\dots l$} \Comment{Możemy sprawdzać tylko parzyste}
	\ForAll{$j \in 0\dots n-i$}
		\State $s \gets S[j \dots i]$
		\If {$s$ jest ciasnym bliźniakiem}
			\State
		    \Return $true$
	    \EndIf
	\EndFor
\EndFor
\State
\Return $false$
\end{algorithmic}
\end{frame}

\begin{frame}

\includegraphics[height=1\textheight]{img/cache-all-the-things.png}


\end{frame}

\begin{frame}
\begin{examples}
$$
1257 \equiv 0123 \equiv 1234
$$
\end{examples}
\end{frame}

\begin{frame}
\frametitle{Znormalizowana sekwencja}
\begin{algorithmic}
\State $S \gets $ sekwencja wejściowa
\State $T \gets \emptyset$
\State $U \gets \emptyset$
\ForAll{$a \in S$}
		\If {$(a, i) \in U$}
			\State
			\Call{Append}{$T, i$}
		\Else
			\Call{Append}{$T, |U|$}
			\State $U \gets U ~ \cup ~ (a, |U|)$
	    \EndIf
\EndFor
\State
\Return T
\end{algorithmic}

\end{frame}


\end{document}


Submit Click To Copy Select All Text
\documentclass{beamer}

\usepackage[utf8]{inputenc}
\usepackage[T1]{fontenc}
\usepackage{amsthm}
\usepackage{amssymb}
\usepackage{amsfonts}
\usepackage{amsmath}
\usepackage[]{algorithmicx}
\usepackage{algpseudocode}% http://ctan.org/pkg/algorithmicx
\usepackage[polish]{babel}
\selectlanguage{polish}
%%% fix for \lll
\let\babellll\lll
\let\lll\relax
\usetheme{Warsaw}

\title{
	Gra w unikanie ciasnych bliźniaków\\
}

\author{
	Jakub Dutkowski \and
	Karol Dzitkowski \and
	Tomasz Janiszewski \and
	Michał Padzik
}

\institute[MiniPW] % (optional)
{
  Wydział Matematyki i Nauk Informacyjnych\\ Politechnika Warszawska
}

\date[VLC 2013] % (optional)
{Kombinatoryka na słowach, Czerwiec 2016}

\logo{\includegraphics[height=1.0cm]{img/mini-logo.png}}

\begin{document}

\frame{\titlepage}
\begin{frame}
	\tableofcontents
\end{frame}

\section{Opis Gry}

\begin{frame}
	\frametitle{Temat}
	\begin{itemize}
		\item Gra komputer kontra człowiek (co najmniej 2 poziomy trudności)
		\item Gra komputer kontra komputer (co najmniej 2 poziomy trudności)
	\end{itemize}
\end{frame}

\begin{frame}
	\frametitle{Definicje}
	\begin{block}{Bliźniaki}\label{def:blizniak}
		Niech
		$S = s_1 s_2 s_3 \ldots\ s_n$
		będzie skonczoną sekwencją symboli. Dwie pod-sekwencje
		$S_1 = s_{i_1} s_{i_2} s_{i_3} \ldots\ s_{i_k}$
		i
		$S_2 = s_{j_1} s_{j_2} s_{j_3} \ldots\ s_{j_k}$
		nazywane są bliźniakami jeśli są identyczne oraz rozłączne, czyli
		$$
		\left\{ \begin{array}{ll}
		s_{i_p} = s_{j_p} &\forall p = {1, 2, 3, \ldots\, k} \\
		i_p \neq j_p &\forall p,q = {1, 2, 3, \ldots\, k}
		\end{array} \right.
		$$
	\end{block}
\end{frame}

\begin{frame}
	\frametitle{Definicje}
	\begin{block}{Ciasne Bliźniaki}\label{def:ciasne_blizniak}
		Mówimy, że para bliźniaków jest ciasna,
		jeżeli połączenie zbiorów ich indeksów tworzy pełny segment
		liczb całkowitych,
		czyli para tworzy jeden blok.
	\end{block}
\end{frame}

\begin{frame}
	\frametitle{Definicje}
	\begin{examples}{Przykład}
		Dla przykładu ciąg
		$23\overline{12}\underline{1}\overline{3}\underline{23}1$
		ma ciasnych bliźniaków
		w formie $123$, natomiast ciąg $123132312$ nie ma żadnych.
	\end{examples}
\end{frame}

\begin{frame}
	\frametitle{Definicje}
	Gra odbywa się na planszy -- słowie. Początkowo nie ma na niej nic.
	Gracz pierwszy wskazuje miejsce w słowie w którym należy dopisać znak.
	Gracz drugi dopisuje znak z alfabetu. Gracz pierwszy wygrywa
	gdy w słowie znajdą się ciasne bliźniaki.
	Gracz drugi wygrywa gdy w gra się zakończy i w słowie nie będzie ciasnych
	bliźniaków.
	Gra kończy się gdy w słowie są ciasne bliźniaki lub
	został osiągnięty limit długości słowa.
\end{frame}

\section{Algorytmy}

\begin{frame}
	\frametitle{Sprawdzanie czy sekwencja zawiera ciasnych bliźniaków}

	\begin{algorithmic}
		\State $S \gets $ sekwencja wejściowa
		\State $n \gets $ długość $S$
		\ForAll{$i \in 0\dots l$}
		\ForAll{$j \in 0\dots n-i$}
		\State $s \gets S[j \dots i]$
		\If {$s$ jest ciasnym bliźniakiem}
		\State
		\Return $true$
		\EndIf
		\EndFor
		\EndFor
		\State
		\Return $false$
	\end{algorithmic}
\end{frame}

\begin{frame}
	\frametitle{Sprawdzanie czy sekwencja jest ciasnym bliźniakiem}

	\begin{algorithmic}
		\State $S \gets $ sekwencja wejściowa
		\State $subsets \gets $ wszystkie równoliczne podziały $S$ na dwie sekwencje
		\ForAll{$(a,b) \in subests$}
		\If {$a = b$}
		\State
		\Return $true$
		\EndIf
		\EndFor
		\State
		\Return $false$
	\end{algorithmic}

\end{frame}

\begin{frame}
	\begin{algorithmic}
		\State $subsets \gets \emptyset$
		\ForAll{$i \in 0\dots 2^n$}
		\State $b \gets $ binarna reprezentacja $i$
		\State $ones \gets \sum b $ \Comment{liczba jedynek w $b$}
		\If {$ones = \frac{n}{2}$}
		\State $A \gets \emptyset$
		\State $B \gets \emptyset$
		\ForAll{$i \in 0 \dots n$}
		\If {$ones[i] = 1$}
		\State
		\Call{Append}{$A, S[i]$}
		\Else
		\State
		\Call{Append}{$B, S[i]$}
		\EndIf
		\EndFor
		\State $subsets \gets subsets ~ \cup ~ (A,B)$
		\EndIf
		\EndFor
		\State
		\Return $subsets$
	\end{algorithmic}

\end{frame}

\subsection{MinMax}

\begin{frame}
	\frametitle{MinMax}

	Gracz pierwszy stara się zakończyć grę z jak najkrótszym słowem.
	Natomiast gracz drugi z jak najdłuższym.

\end{frame}

\begin{frame}
	\frametitle{MinMax}

	\begin{algorithmic}
		\Function{MaxMove}{$S$}
		\State $r \gets -\infty$
		\ForAll{możliwy ruch}
		\State $m \gets -\infty$
		\If {ruch nie powoduje powstanie ciasnych bliźniaków}
		\If {osiągnięto limit słowa}
		\State $m \gets +\infty$ \Comment{Wygrywamy}
		\Else
		\State $m \gets $ \Call{MinMove}{plansza po ruchu}
		\EndIf
		\Else
		\State $m \gets $ długość słowa \Comment{Chcemy grać jak najdłużej}
		\EndIf
		\State $r \gets $ \Call{$max$}{$r, m$}
		\EndFor
		\State
		\Return ruch z wartością $r$
		\EndFunction
	\end{algorithmic}

\end{frame}

\begin{frame}
	\frametitle{MinMax}

	\begin{algorithmic}
		\Function{MinMove}{$S$}
		\State $r \gets +\infty$
		\ForAll{możliwy ruch}
		\State $m \gets +\infty$
		\If {ruch powoduje powstanie ciasnych bliźniaków}
		\If {osiągnięto limit słowa}
		\State $m \gets -\infty$ \Comment{Wygrywamy}
		\Else
		\State $m \gets $ \Call{MaxMove}{plansza po ruchu}
		\EndIf
		\Else
		\State $m \gets $ długość słowa \Comment{Chcemy grać jak najkrócej}
		\EndIf
		\State $r \gets $ \Call{$min$}{$r, m$}
		\EndFor
		\State
		\Return ruch z wartością $r$
		\EndFunction
	\end{algorithmic}

\end{frame}

\section{Optymalizacje}

\begin{frame}
	\frametitle{Sprawdzanie czy sekwencja zawiera ciasnych bliźniaków}

	\begin{algorithmic}
		\State $S \gets $ sekwencja wejściowa
		\State $n \gets $ długość $S$
		\ForAll{$i \in 0\dots l$} \Comment{Możemy sprawdzać tylko parzyste}
		\ForAll{$j \in 0\dots n-i$}
		\State $s \gets S[j \dots i]$
		\If {$s$ jest ciasnym bliźniakiem}
		\State
		\Return $true$
		\EndIf
		\EndFor
		\EndFor
		\State
		\Return $false$
	\end{algorithmic}
\end{frame}

\begin{frame}

	\includegraphics[height=1\textheight]{img/cache-all-the-things.png}


\end{frame}

\begin{frame}
	\begin{examples}
		$$
		1257 \equiv 0123 \equiv 1234
		$$
	\end{examples}
\end{frame}

\begin{frame}
	\frametitle{Znormalizowana sekwencja}
	\begin{algorithmic}
		\State $S \gets $ sekwencja wejściowa
		\State $T \gets \emptyset$
		\State $U \gets \emptyset$
		\ForAll{$a \in S$}
		\If {$(a, i) \in U$}
		\State
		\Call{Append}{$T, i$}
		\Else
		\Call{Append}{$T, |U|$}
		\State $U \gets U ~ \cup ~ (a, |U|)$
		\EndIf
		\EndFor
		\State
		\Return T
	\end{algorithmic}

\end{frame}


\end{document}
