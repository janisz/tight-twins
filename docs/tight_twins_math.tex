\documentclass[11pt,a4paper]{article}

\usepackage[polish]{babel}
\usepackage[utf8]{inputenc}
\usepackage[T1]{fontenc}
\usepackage{lmodern}
\usepackage{indentfirst}
\usepackage{fullpage}
\usepackage{enumerate}
\usepackage{graphicx}
\usepackage{authblk}
\usepackage{hyperref}
\usepackage{amsthm}
\usepackage{amsfonts}
\usepackage{amsmath}

\theoremstyle{definition}
\newtheorem{definition}{Definition}[section]
\newtheorem{theorem}{Theorem}[section]
\newtheorem{lemma}{Lemma}[section]
\newtheorem{example}{Example}[section]

\selectlanguage{polish}

% Zdefiniowanie autorów i~tytułu:
\author{Jakub Dutkowski}
\author{Karol Dzitkowski}
\author{Tomasz Janiszewski}
\author{Michał Padzik}
\affil{Wydział Matematyki i Nauk Informacyjnych, Politechnika Warszawska}
\title{
	Kombinatoryka na Słowach\\
	Gra w unikanie ciasnych bliźniaków\\
	Opis matematyczny projektu
	6bc\footnote{Gra komputer kontra człowiek (co najmniej 2 poziomy trudności)}
\footnote{Gra komputer kontra komputer (co najmniej 2 poziomy trudności)}
 }

 \renewcommand\Authands{ i }

\begin{document}
\maketitle\
\newpage\

\section{Podstawy teoretyczne pracy}
%Treść:
\begin{definition}{(kwadrat)}\label{def:sqr-word}
Słowo $w$, które zawiera podsłowo dwa razy z rzędu
nazywamy $słowem\;kwadratowym$
\end{definition}
\begin{definition}{(square-free word)}\label{def:sqr-free-word}
Słowo $w$, które nie zawiera żadnego podsłowa dwa razy z rzędu
nazywamy $słowem\;niekwadratowym$
\end{definition}
\begin{definition}{(Bliźniak)}\label{def:blizniak}
Niech $S = s_1 s_2 s_3 \ldots\ s_n$ będzie skończoną sekwencją symboli. Dwie pod-sekwencje
$S_1 = s_{i_1} s_{i_2} s_{i_3} \ldots\ s_{i_k}$ i $S_2 = s_{j_1} s_{j_2} s_{j_3} \ldots\ s_{j_k}$
nazywane są bliźniakami jeśli są identyczne oraz rozłączne, czyli $s_{i_p} = s_{j_p}$ dla
wszystkich $p = {1, 2, 3, \ldots\, k}$ oraz $ i_p \neq j_p $ dla wszystkich $ p,q = {1, 2, 3, \ldots\, k}$.
\end{definition}
\begin{definition}{(Ciasny bliźniak)}
Mówimy, że para bliźniaków jest ciasna, jeżeli połączenie zbiorów ich indeksów tworzy pełny segment
liczb całkowitych, czyli para tworzy jeden blok. Dla przykładu ciąg $231213231$ ma ciasnych bliźniaków
w formie $123$, natomiast ciąg $123132312$ nie ma żadnych.
\end{definition}
\begin{definition}[($k-triola$)]\label{def:triola}
$k-triolą$ nazywamy sekwencję zawierającą $k$ identycznych subsekwencji
których indeksy są parami rozłączne.
\end{definition}
\begin{definition}[($k-wolną-triola$)]\label{def:triola}
$k-wolną-triolą$ nazywamy sekwencję, której żaden segment
nie jest $k-triolą$.
\end{definition}

\begin{lemma}\label{lemat:1}
Słowo które zawiera pokrywające się podciągi jest kwadratowe
\begin{proof}
Niech $w$ będzie słowem zawierającym dwa pokrywające się podciągi tego
samego niepustego słowa $u$.
Wtedy $$w = zuy = x'uy'$$ dla słów $x,x',y,y'$. Możemy założyć że $x$
jest krótsze niż $x'$. Ponieważ wystąpienia się pokrywają,
jedno ma długość $|x| < |x'| < |xu| < |x'u|$. Więc podstawiając
$sx = x'$ i $x'q = xu$ otrzymujemy $xu = x'q = xsq$
skąd $u =  sq$ i $$w = x'uy' = xssqy' $$
Pokazując że $w$ zawiera kwadrat $ss$.
\end{proof}
\end{lemma}

\begin{lemma}\label{lemat:2}
Niech $p$ będzie słowem takim że $p^2$ nie zawiera żadnego innego kwadratu niż $p$.
$\forall n > 1$ jeśli
$p^n$ zawiera $u^2$, wtedy $$|u| \equiv 0 ~mod~ |p|^3$$
\begin{proof}
Niech $u^2$ będzie częścią $p^n$. Pokażemy że istnieje prefiks $x$ i $x'$
słowa $p$ i słowa $y,y'$ oraz $k \in \mathbf{Z}$ takie że
$$p^k = zuy = x'uy'$$
Przyjmijmy że $|x| < |x'|$. Jeśli $xu$ jest krótsze niż $x'$, to pierwsze wystąpienie
$u$ jest częścią $p$. Ale wtedy $u^2$ jest częścią $p^2$. Więc te dwa wystąpienia $u$
zachodzą na siebie w $p^k$.
Podstawiając $xs = x'$ lemat \ref{lemat:1} pokazuje że $s^2$ jest częścią $p^2$.
Co dowodzi
$$ s = p $$
\end{proof}
\end{lemma}

\begin{theorem}
Niech $n > 0 \in \mathsf{N}$ i
niech $L_1,L_2,\ldots,L_n$ będzie sekwencją
siedmioelementowych zbiorów. Wtedy istnieje sekwencja $S=s_1s_2\ldots s_n$ w której
$\forall s_i \in L_i \quad i = 1,2,\ldots,n$ nie ma bliźniaków\ref{def:blizniak}
\end{theorem}

\begin{theorem}
Dla każdego $k$ istnieje funkcja $f(k)$, taka że
dla dowolnej sekwencji list, każda o długości $f(k)$
możliwe jest wybranie $k-wolnej-trioli$ z tych list.
\end{theorem}

\begin{theorem}
Nad czteroliterowym alfabetem istnieje $niekwadratowe$
słowo dowolnej długości.
\begin{proof}
Pokażemy, że dowolne niekwadratowe słowo długości $k$ nad alfabetem $A$
długości $4$, może zbudować dłuższe słowo nad alfabetem $A$.

Niech $p$ będzie słowem and alfabetem trzy literowym -- na przykład $a,b,c$.
Niech $p$ będzie miało długość co najmniej $4$.
Niech $p^2$ nie zawiera innego kwadratu niż ono samo.
Po przez wprowadzenie nowej litery -- na przykład $d$, pomiędzy
dwie litery w $p$ w czterech różnych miejscach, otrzymamy cztery
słowa $x,y,z,t$ z których każde zawiera tylko jedno wystąpienie litery $d$,
a które połączone sprowadzają się do $p$ po przez usunięcie nowej litery -- $d$.

\begin{example}
Zaczynając od $p = abacbc$ możemy otrzymać
$$
x = adbacdc, y = abdacbc, z = acadcbc, t = abacdbc
$$
zdefiniujmy przekształcenie
$$
h: \left\lbrace a,b,c,d\right\rbrace\star \rightarrow \left\lbrace a,b,c,d\right\rbrace\star
$$
przez
$$
h(a) = x, h(b) = y, h(c) = z, h(d) = t
$$
Należy udowodnić że $h$ jest przekształceniem niekwadratowym -- na przykład
$h(u)$ jest niekwadratowe jeśli $u$ też jest niekwadratowe.
\end{example}
Załóżmy że $u$ jest niekwadratowe. Niech $w=h(u)$ gdzie $h$ jest przekształceniem
zdefiniowanym wyżej. Załóżmy że $w$ zawiera $v^2$. Wtedy
$$
w = h(u) = \alpha v^2\beta
$$
Niech $v'$ pochodzi z $v$ poprzez usunięcie wystąpień litery $d$.

$v$ zawiera co najmniej jedno wystąpienie $d$. W przeciwnym razie $v = v'$ i
$v'^2$ jest częścią $w$ a w konsekwencji $p^2$ (wbrew założeniom). Korzystając z
powyższych lematów wyprowadźmy
$$
v' = p_2 p^l P_1 m \quad l \geq 0, p = p_1 p_2
$$
Więc $v$ zawiera dokładnie $1+l$ wystąpień litery $d$. Niech
$$
v = sr_1\dots r_l\overline{s} = s'r'_1\dots r'_l\overline{s'}
$$
gdzie
$$
r_1,\ldots,r_l,r_1',\ldots,r_l',\overline{s},\overline{s}'
\in X = \left\lbrace x,y,z,t \right\rbrace
$$
Jeśli $s \neq s'$ wtedy łatwo zauważyć że $p^2$ zawiera kwadrat.
Zatem $s=s', r_i=r_i'$ dla
$1 \leq i \leq l$
i
$\overline{s} = \overline{s}'$. Ponieważ $\overline{s}s'$ zawiera
jedno $d$ s (i $s'$) lub $\overline{s}$ (and $\overline{s}'$)
zawierają $d$. Ale sufiks/prefiks słowa w $X$ zawierającego $d$
determinuje słowo w $X$. To oznacza że $u$ zawiera kwadrat.
\end{proof}
\end{theorem}

\begin{theorem}
Istnieje co najmniej $5.59^n$ słów długości $n$ nad alfabetem długości $7$ nie zawierających ciasnych bliźniaków.
\begin{proof}
Niech $\Sigma_m=\{0,1,\cdots ,m-1\}$ będzie alfabetem złożonym z m znaków i niech $L \subset \Sigma^{*}_m$ będzie językiem złożonym z zabronionych podciągów o długości co najmniej 2. Oznaczmy złożoność podciągu $L$ przez $n_i = L \cap \Sigma^{i}_m$. Definiujemy $L'$ jako zbiór takich słów $w$, że $w$ nie należy do $L$ i jego prefix o długości $|w| - 1$ słowa $w$ należy do $L$. Dla każdego zabronionego podciągu $f \in F$, wybieramy liczbę $1 \leq s_f \leq |f|$. Wtedy dla każego $i \geq 1$, definiujemy $a_i \in \mathbb{Z}$ takie, że 
\begin{equation} \label{eq:1}
	a_i \geq \max_{u \in L}|\{v \in \Sigma^{i}_m | uv = bf, f \in F, s_f = i \}|
\end{equation}

Rozważmy formalny szereg potęgowy $P(x) = 1 - mx + \Sigma_{i\geq 1} a_i x^i$. Jeżeli $P(x)$ ma dodatni rzeczywisty pierwiastek $x_0$, to $n_i \geq x_0^{i-1}$ dla każdego $i\geq 0$.

Przekształćmy $P(x) = 1 - mx_0 + \Sigma_{i\geq 1} a_i x_0^i = 0$ jako

\begin{equation} \label{eq:2}
m - \Sigma_{i\geq 1} a_i x_0^{i-1} = x_0^{-1}
\end{equation}

Jako, że $n_0=1$, wykażemy przez indukcję, że $\frac{n_i}{n_{i-1}} \geq x_0^{-1}$ po to, by pokazać, że $n_i \geq x_0^{-i}$ dla każdego $i \geq 0$. Używając \autoref{eq:2}, widać że w przypadku początkowym $\frac{n_1}{n_0} = n_1 = m \geq x_0^{-1}$. Idąc dalej, dla każdego $i \geq 1$ jest:
\begin{itemize}
\item $m^i słów w \Sigma^{i}_m$
\item $n^i słów w L$
\item co najwyżej $\Sigma_{1\leq j\leq i} n_{i-j} a_j$ słów w $L'$
\item $m(m^{i-1} - n_{i-1})$ słów w $\Sigma^{i}_m \setminus {L \cup L'}$
\end{itemize}

Stąd widać, że $n_i + \Sigma_{1\leq j\leq i} n_j a_{i-j} + m(m^{i-1} - n_{i-1}) \geq m^i$, więc $n_i \geq mn_{i-1} - \Sigma_{1\leq j\leq i} n_{i-j} a_j$.

\begin{equation}
\begin{split}
\frac{n_i}{n_{i=1}} & \geq m - \Sigma_{1\leq j\leq i} a_{j} \frac{n_{i-j}}{n_{i-1}} \\
& \geq m - \Sigma_{1\leq j\leq i} a_{j} x_0^{j-1} \\
& \geq m - \Sigma_{j \geq 1} a_{j} x_0^{j-1} \\
& = x_0^{-1}
\end{split}
\end{equation}

Teraz pokażemy jak za pomocą zdefiniowanej metody używać do unikania ciasnych bliźniaków. \textit{q-prefiksem} (odp. \textit{1-sufiksem}) słowa nazywamy jego prefiks (odp. sufiks) o długości $q$. Ciasny bliźniak nazywany jest \textit{minimalnym}, jeśli nie zawiera krótszych ciasnych bliźniaków jako podciągów. Ciasny bliźniak jest \textit{mały}, jeśli jego długość to 2 i \textit{duży} jeśli jest dłuższy. Zbiór $F$ zabronionych podciągów zawiera każdy minimalny ciasny bliźniak. Definiujemy $s_f 1$ jeśli $f \in F$ jest mały i $s_f = |f| - 2$ jeśli nie jest.
\\
Ustalamy $a_1 = 1$, ponieważ $s_f=1$ tylko dla małych ciasnych bliźniaków i istnieje tylko jeden sposób na rozszerzenie prefiksu o jedną literę tak, aby uzyskać sufiks $xx$ z $x \in \Sigma_m$. Aby uzyskać rozsądną górną granicę $a_t$ dla $t \geq 2$, musimy ograniczyć liczbę dużych minimalnych ciasnych bliźniaków. Dla każdego ciasnego bliźniaka $f$ słowa $w$ o długości $i$, definiujemy $funkcję \; wysokości \; h: \; [0,\cdots,2i] \rightarrow \mathbb{Z}$ zdefiniowaną następująco:
\begin{itemize}
\item $h(0) = 0$
\item Dla $0\leq j \leq 2i, h(j) = h(j-1) + 1$ jeśli j-ta litera $f$ należy do podsłowa $w$ zawierającego pierwszą literę $f$ i $h(j) = h(j-1) -1$ jeśli jest przeciwnie.
\end{itemize}

Jako, że $f$ jest ciasnym bliźniakiem, mamy $h(2i) = 0)$. Co więcej, jeśli $h(j) = 0$ dla niektórych $0 \leq j \leq 2i$, wtedy prefiks długości $j$ słowa $f$ jest ciasnym bliźniakiem. Wiec, jeśli $h$ jest wartością funkcji wysokości minimalnego ciasnego bliźniaka, wtedy $h(j)>0$ dla każdego $0 \leq j \leq 2i$. Stąd wysokość każdego minimalnego bliźniaka jest związana z unikalnym słowem Dyck'a o długości $2i-2$. Stąd, liczba funkcji wysokości jest co najwyżej równa $\frac{(2i - 2)!}{i!(i-1)!}$. Zgodnie z \autoref{eq:1} musimy ograniczyć liczbę rozwiązań $uv=bf$ takich, że $u$ jest ustalone i $|v| = s_f = |f| -2 = 2i - 2$. 2-prefiks słowa $f$ jest ustalony, jako że odpowiada 2-sufiksowi u. Zauważmy, że 2-prefiks dużego minimalnego ciasnego bliźniaka słowa $w$ jest równy 2-prefiksowi $w$, więc 2-prefiks $w$ także jest ustalony. Stąd, jest co najwyżej $m^{i-2}$ możliwości $w$. Jako, że $f$ jest wyznaczone przez $w$ i jego funkcję wysokości, istnieje co najwyżej $m^(i-2)\frac{(2i - 2)!}{i!(i-1)!}$ możliwości słowa $f$. Ustawmy więc $a_{2i-2} = m^{i-2}\frac{(2i - 2)!}{i!(i-1)!}$ i rozważmy wielomian
\begin{equation}
\begin{split}
P(x) & = 1 - mx + x + \Sigma_{i \geq 2} m^(i-2)\frac{(2i - 2)!}{i!(i-1)!} x^{2i-2} \\
& = 1 - (m-1)x + \Big(\frac{2x}{1+ \sqrt{1-4mx^2}} \Big)^2
\end{split}
\end{equation}

Dla $m=6$, $P(x)$ nie ma dodatniego pierwiastka. Dla m=7, mamy $P(x_0)=0$ z $x_0=0.1788487593\cdots$. Więc istnieje co najmniej $\alpha^b$ słow długości $n$ nad $\Sigma_7$, które nie zawierają ciasnych bliźniaków, gdzie $\alpha = x_0^{-1} = 5.5913163944\cdots$.

\end{proof}
\end{theorem}

\begin{thebibliography}{4}
%
\bibitem{grytczuk}
J. Grytczuk, J. Kozik and B. Zaleski
\textsl{Avoiding tight twins in sequences via
entropy compression}
\bibitem{short_proof}
Guillaume Guegan and Pascal Ochem
\textsl{A short proof that shuffle squares are
7-avoidable}
\bibitem{thue}
J. Berstel
\textsl{Axel Thueís papers on repetitions in words: a translation}
\end{thebibliography}
\end{document}