\documentclass[11pt,a4paper]{article}

\usepackage[polish]{babel}
\usepackage[utf8]{inputenc}
\usepackage[T1]{fontenc}
\usepackage{lmodern}
\usepackage{indentfirst}
\usepackage{fullpage}
\usepackage{enumerate}
\usepackage{graphicx}
\usepackage{authblk}

\selectlanguage{polish}

% Zdefiniowanie autorów i~tytułu:
\author{Jakub Dutkowski}
\author{Karol Dzitkowski}
\author{Tomasz Janiszewski}
\author{Michał Padzik}
\affil{Wydział Matematyki i Nauk Informacyjnych, Politechnika Warszawska}
\title{
	Kombinatoryka na Słowach\\
	Gra w unikanie ciasnych bliźniaków\\
	Teoretyczny opis problemu
	6bc\footnote{Gra komputer kontra człowiek (co najmniej 2 poziomy trudności)}
\footnote{Gra komputer kontra komputer (co najmniej 2 poziomy trudności)}
 }
 
 \renewcommand\Authands{ i }

\begin{document}
\maketitle\
\newpage\


%Treść:
\section{Definicja ciasnych bliźniaków}
Niech $S = s_1 s_2 s_3 \ldots\ s_n$ będzie skończoną sekwencją symboli. Dwie pod-sekwencje
$S_1 = s_{i_1} s_{i_2} s_{i_3} \ldots\ s_{i_k}$ i $S_2 = s_{j_1} s_{j_2} s_{j_3} \ldots\ s_{j_k}$
nazywane są bliźniakami jeśli są identyczne oraz rozłączne, czyli $s_{i_p} = s_{j_p}$ dla 
wszystkich $p = {1, 2, 3, \ldots\, k}$ oraz $ i_p \neq j_p $ dla wszystkich $ p,q = {1, 2, 3, \ldots\, k}$.
Mówimy że para bliźniaków jest ciasna, jeżeli połączenie zbiorów ich indeksów tworzy pełny segment
liczb całkowitych, czyli para tworzy jeden blok. Dla przykładu ciąg $231213231$ ma ciasnych bliźniaków
w formie $123$, natomiast ciąg $123132312$ nie ma żadnych.

\section{Temat projektu}
Projekt będzie realizował grę w unikanie ciasnych bliźniaków,
w której biorą udział 2 osoby 
\begin{enumerate}
	\item\ Gracz Pierwszy (A) [sterowany przez maszynę]
	\item\ Gracz Drugi (B) [sterowany przez człowieka]
\end{enumerate}
Planszą jest słowo $s$, które składa się ze skończonej,
liczby znaków $n$ nad skończonym alfabetem $A$.
Gracz A wskazuje miejsce pomiędzy symbolami w słowie $s$, w którym gracz B
wstawia dowolny symbol z alfabetu $A$.
Zadaniem gracza A jest zmuszenie gracza B do stworzenia 
słowa zawierającego ciasnych bliźniaków.

\section{Opis problemu}

\end{document}